\documentclass[12pt]{article}
\usepackage{mathtools}
\addtolength{\textheight}{.5in}
\addtolength{\textwidth}{1in}
\addtolength{\topmargin}{-.25in}
\addtolength{\evensidemargin}{-.5in}
\addtolength{\oddsidemargin}{-.5in}
\usepackage{placeins}
\begin{document}

\section{Scale factor as a function of time}
From the Friedmann Equation we have
$$ (\frac{\dot{a}}{a})^{2} = \frac{8\pi G}{3}[\rho_{m} + \rho_k + \rho_r + \rho_{\Lambda}] $$
$$ \frac{\dot{a}}{a} = (\frac{8\pi G}{3})^{1/2}[\rho_{m} + \rho_k + \rho_r + \rho_{\Lambda}]^{1/2}$$

In homework 1 we worked out that the ith species $\rho_i = \rho_{0 i}a^{\alpha}$ for alpha equal to -4, -3, -2, 0, 2 for relativistic, non-relativistic, lambda, and curvature matter respectively. If we take a flat model universes with only relativistic or non-relativistic matter, we can solve the Friedmann Equation in the following way

$$ \frac{\dot{a}}{a} = (\frac{8\pi G}{3})^{1/2}[\rho_{i}(a)]^{1/2} $$
$$ \frac{\dot{a}}{a} = (\frac{8\pi G \rho_{0 i}}{3})^{1/2}a^{\alpha/2} $$
$$ \frac{da}{dt} \frac{1}{a} = (\frac{8\pi G \rho_{0 i}}{3})^{1/2}a^{\alpha/2} $$
$$ da \frac{1}{a} a^{-\alpha/2}= (\frac{8\pi G \rho_{0 i}}{3})^{1/2} dt $$
$$ da  a^{-\alpha/2 -1}= (\frac{8\pi G \rho_{0 i}}{3})^{1/2} dt $$
$$ a^{-\alpha/2} \frac{1}{-\alpha/2} = (\frac{8\pi G \rho_{0 i}}{3})^{1/2} t + C$$
$$ a(t)  = [\frac{-t\alpha}{2} (\frac{8\pi G \rho_{0 i}}{3})^{1/2}  -C \alpha/2]^{-2/\alpha} $$

The initial condition $a(0) = 0$ requires that the integration constant is zero. The other condition $a_0 = a(t_0)$ can be used to show
$$a_0  = [\frac{-t_0 \alpha}{2} (\frac{8\pi G \rho_{0 i}}{3})^{1/2}]^{-2/\alpha}$$
$$a_{0}^{-\alpha/2} \frac{2}{-t_0 \alpha} = (\frac{8\pi G \rho_{0 i}}{3})^{1/2}$$

Finally, for a relativistic matter only universe we have alpha=-4, which gives us the solution
$$a_{r}(t)  = (\frac{t}{t_0 })^{1/2} a_{0}$$

For non-relativistic matter only universe we have alpha = -3, which gives us the solution
$$ a_{m}(t)  = (\frac{t}{t_0})^{2/3} a_0$$

The Friedmann equation for a $\rho_{\Lambda}$ universe will lead us to

$$ da \frac{1}{a} = (\frac{8\pi G \rho_{\Lambda}}{3})^{1/2} dt $$
$$ \ln a =  (\frac{8\pi G \rho_{\Lambda}}{3})^{1/2} t + C$$
$$a(t) = \exp[(\frac{8\pi G \rho_{\Lambda}}{3})^{1/2} t]  C'$$

Enforcing the condition $a(t_0) = a_0$ gives us 
$$a_0 = \exp[(\frac{8\pi G \rho_{\Lambda}}{3})^{1/2} t_0]  C'$$
$$\frac{\ln a_0/C'}{t_0} = (\frac{8\pi G \rho_{\Lambda}}{3})^{1/2} $$

$$a(t) = \exp[\ln (a_0/C') t/ t_{0}]  C'$$

\section{Equation of state}
If we parameterize the equation of state for dark energy with $w(a) = w_0 + w_{a}(1 - a)$, the dark energy density becomes

$$ \omega_{Q} = \omega_{Q 0}a^{-3(1 + w(a))}$$
$$ \omega_{Q} = \omega_{Q 0}a^{-3(1 + w_0 + w_{a}(1 - a))}$$

\section{Homogenius Scalar Field}
\subsection{Equation of motion}
The equation of motion is $\ddot{\phi} + 3H\dot{\phi} + V'(\phi) = 0$. We can rewrite H by using the Friedmann equation $\frac{8 \pi G}{3} (\rho_m + \rho_{\phi}) = H^2$.  Additionally, our potential $V(\phi) = V_{0}e^{-\lambda \phi}$, which we use to evaluate $V'(\phi) = \frac{\partial V}{\partial \phi} = -\lambda V_{0}e^{-\lambda \phi} = -\lambda V(\phi)$. Throwing all of this together, the equation of motion becomes

$$\ddot{\phi} + 3\Big[ \frac{8 \pi G}{3} \rho_{tot} \Big]^{1/2}\dot{\phi} - \lambda V(\phi) = 0$$

Now from the stress energy tensor, we have $\rho_{\phi} = \dot{\phi}^2 + V(\phi)$. If we require $V(\phi) = \frac{1}{2} \rho_{\phi}$, we can use the stress energy tensor relation to get the relationship $\rho_{\phi} = \dot{\phi}^2$. Now the equation of motion is 

$$\ddot{\phi} + 3\Big[ \frac{8 \pi G}{3} \rho_{tot} \rho_{\phi} \Big]^{1/2} - \frac{\lambda}{2} \rho_{\phi} = 0$$

Keeping an eye towards what is to come in the next question, we divide through by $\rho_{\phi} $ so that the term $\rho_{tot}/ \rho_{\phi} $ crops up. We then use $\rho_{\phi} = \dot{\phi}^2$ once more. Now we assert if the matter density is always a fixed fraction of the scalar field density, the fraction $\rho_{tot}/ \rho_{\phi}$ is fixed for all times. Now we have an equation in terms of $\phi$, time, and constants only.

 $$\ddot{\phi} \frac{1}{\dot{\phi}^2} + 3\Big[ \frac{8 \pi G}{3} \frac{\rho_{tot}}{ \rho_{\phi}} \Big]^{1/2} - \frac{\lambda}{2} = 0$$
 
 For the sake of readabiltiy, let us define $3\Big[ \frac{8 \pi G}{3} \frac{\rho_{tot}}{ \rho_{\phi}} \Big]^{1/2} - \frac{\lambda}{2} = \kappa $. To tackle the differential equation, let us re-write the equation of motion as a first order equation. We define $f = \dot{\phi}, \dot{f} = \ddot{\phi}$. Now we slide things around, separate variables, and integrate. 
 
 $$\frac{d f}{d t} = -\kappa f^2 $$
 $$ \frac{df}{f^2} = -\kappa dt$$
 $$ \frac{1}{f} = \kappa t + C_1 $$
 $$ f = \frac{1}{\kappa t - C_1} $$
 $$ d\phi = \frac{1}{\kappa t - C_1} dt$$
 $$ \phi(t) = \frac{1}{\kappa} \ln(\kappa t - C_1) + C_2 $$
 
Where $C_1, C_2$ are integration constants. Now we have shown a solution exists for the EOM given the assumptions in the problem.

\subsection{$\rho_{\phi}/\rho_{tot}$ in terms of $\lambda$}

We can eliminate the integration constants from our solution for the scalar field by using the relationship $\rho_{\phi} = 2V(\phi) = \dot{\phi}^2$

$$ \frac{d\phi}{dt} = \sqrt{2V_0}e^{-\lambda \phi/2} $$
$$ d\phi e^{\lambda \phi/2}= \sqrt{2V_0} dt$$
$$  \frac{2}{\lambda} e^{\lambda \phi/2}= \sqrt{2V_0} t + C_3$$
$$ e^{\lambda \phi/2}= \lambda \sqrt{V_{0}/2} t + \frac{\lambda}{2}C_3$$
$$ \phi(t)= \frac{2}{\lambda} \ln  \Big(\lambda \sqrt{V_{0}/2} t + \frac{\lambda}{2}C_3 \Big)$$

Now at time = 0 and $\phi(0) = \phi_0$ we have
$$ e^{\lambda \phi_{0}/2} =  \frac{\lambda}{2}C_3$$
$$ \frac{2}{\lambda} e^{\lambda \phi_{0}/2} = C_3$$
$$ \phi(t)= \frac{2}{\lambda} \ln  \Big(\lambda \sqrt{V_{0}/2} t + e^{\lambda \phi_{0}/2} \Big)$$

Comparing terms of this solution to our previous solution of $\phi$ which solved the EOM, we must conclude $\frac{2}{\lambda} = \frac{1}{\kappa}$

$$ \frac{\lambda}{2}  = 3\Big[ \frac{8 \pi G}{3} \frac{\rho_{tot}}{ \rho_{\phi}} \Big]^{1/2} - \frac{\lambda}{2} $$
$$ \lambda^2 = 24 \pi G \frac{\rho_{tot}}{ \rho_{\phi}}$$
$$ \frac{24\pi G}{\lambda ^2}  = \frac{\rho_{\phi}}{ \rho_{tot}}$$

\subsection{limits on $\lambda$}

In order to keep the appropriate form of the potential, $\lambda$ must be non negative and non zero. Additionally, if this model universe contains matter in addition to the scalar field, the ratio  $\frac{\rho_{\phi}}{ \rho_{tot}} < 1$. All this together tells us that

$$  \lambda > \sqrt{24 \pi G} $$ 
%\begin{figure}
%\centering
%\includegraphics[width=6in]{Omega_k_neg.png}
%\caption{using a curvature density with negative initial condition}
%\end{figure}

\end{document}